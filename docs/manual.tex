\documentclass[12pt]{report}
\usepackage[utf8]{inputenc}
\usepackage{siunitx}
\usepackage{listings}
\usepackage{hyperref}
\hypersetup{
    colorlinks=false,
}

\newcommand{\packagename}[1]{\textit{#1}}

\title{
	{Writing Physical Computations in Haskell with \packagename{dimensional-dk}}\\
	{\small{Version 0.9}}
}
\author{Björn Buckwalter and Douglas McClean}
\date{\today}

\begin{document}

\maketitle

\tableofcontents

\chapter{Introduction}

\section{Installation}

The latest version of \packagename{dimensional-dk} can be installed in the ordinary way, using the following commands:

\begin{lstlisting}[language=bash]
cabal update
cabal install dimensional-dk
\end{lstlisting}

Historical versions and Haddock documentation are available at \url{http://hackage.haskell.org/package/dimensional-dk}.

\section{Licensing}

The \packagename{dimensional-dk} package is provided under a BSD 3-clause license.

\section{Contributing}

Contributions to \packagename{dimensional-dk} development, in the form of bug reports, feature requests, or pull requests, is welcome.
The source code repository and issue tracker are both hosted by GitHub at \url{https://github.com/bjornbm/dimensional-dk}.

\chapter{Dimensional Arithmetic}

\chapter{Manipulating Units}

\section{Unit Names}

\subsection{Interchange Names}
\subsection{Unit Name Normal Forms}

\chapter{Interoperating with Other Packages}

\section{\packagename{ad} for Automatic Differentiation}
\section{\packagename{attoparsec} for Unit and Quantity Parsing}
\section{\packagename{HaTeX} for \LaTeX{}\. Generation}
\section{\packagename{lens}}
\section{\packagename{linear} for Linear Algebra}
\section{\packagename{parsec} for Unit and Quantity Parsing}
\section{\packagename{time} for Times and Dates}
\section{\packagename{vector} for Boxed and Unboxed Arrays}
\section{\packagename{vector-space} for Linear Algebra}

\chapter{Libraries Using \packagename{dimensional-dk}}

\section{\packagename{atmos} for Modeling Earth's Atmosphere}
\section{\packagename{igrf} for Modeling Earth's Magnetic Field}

\chapter{Included Dimensions and Units}

\end{document}
