\documentclass[12pt]{report}
\usepackage[utf8]{inputenc}
\usepackage{siunitx}
\usepackage{listings}
\usepackage{hyperref}
\hypersetup{
    colorlinks=false,
}

\newcommand{\packagename}[1]{\textit{#1}}
\newcommand{\thispackage}{\packagename{dimensional-dk}}
\newcommand{\experimentalpackage}{\packagename{dimensional-dk-experimental}}
\newcommand{\classname}[1]{\textit{#1}}
\newcommand{\typename}[1]{\textit{#1}}
\newcommand{\modulename}[1]{\textit{#1}}
\newcommand{\submodule}[1]{\modulename{Numeric.Units.Dimensional.DK.{#1}}}

\title{
	{Writing Physical Computations in Haskell with \thispackage}\\
	{\small{Version 0.9}}
}
\author{Björn Buckwalter and Douglas McClean}
\date{\today}

\begin{document}

\maketitle

\tableofcontents

\chapter{Introduction}

The \thispackage library provides data types for performing arithmetic with physical
quantities. Information about the physical dimensions of each quantity is embedded in
its type.

This allows the compiler to statically verify that no dimensional mistakes have been
made in a calculation. It also facilitates interconversion between quantities expressed
in different units and provides documentation of dimension and units of each physical
quantity a program manipulates.

\section{Installation}

The latest version of \thispackage can be installed in the usual way, using the following commands:

\begin{lstlisting}[language=bash]
cabal update
cabal install dimensional-dk
\end{lstlisting}

Historical versions and Haddock documentation are available at \url{http://hackage.haskell.org/package/dimensional-dk}.

\section{Licensing}

The \thispackage package is provided under a BSD 3-clause license.

\section{Contributing}

Contributions to \thispackage development, in the form of bug reports,
feature requests, or pull requests, are welcome. The source code repository and issue
tracker are both hosted by GitHub at \url{https://github.com/bjornbm/dimensional-dk}.

\chapter{Dimensional Arithmetic}

\chapter{Manipulating Units}

\section{Unit Names}

\subsection{Interchange Names}
\subsection{Unit Name Normal Forms}

\chapter{Interoperating with Other Packages}

\section{\packagename{ad} for Automatic Differentiation}
\section{\packagename{attoparsec} for Unit and Quantity Parsing}
\section{\packagename{HaTeX} for \LaTeX{} Generation}
\section{\packagename{ihaskell} for Interactive Notebooks}

The IHaskell project provides support for interactive notebook documents containing
Haskell code through the Jupyter project.

Instances of \classname{IHaskellDisplay} for \typename{Quantity} and \typename{Unit}
allow attractive \LaTeX{}-style display of dimensional values in your notebooks. These
instances can be found in the \submodule{IHaskell} module of the \experimentalpackage{}
package.

\section{\packagename{lens}}

Todo: Without taking a dependency on lens, offer a declaration that can convert a unit into a lens from quantities to numeric values.
Explain here how to use that facility.

\section{\packagename{linear} for Linear Algebra}

Todo: create instances for linear's type classes, recognizing that they require the nasty Functor orphan instance.

\section{\packagename{parsec} for Unit and Quantity Parsing}
\section{\packagename{time} for Times and Dates}

Users of the \packagename{time} package can find functions and lenses for converting
between the \typename{DiffTime} representation and \thispackage{}'s \typename{Time} representation in the
\submodule{Time} module of the \experimentalpackage{} package.

\section{\packagename{vector} for Boxed and Unboxed Arrays}

The \thispackage{} package provides instances of the \typename{Vector} and \typename{MVector} data families and the \classname{Unbox} class
allowing \typename{Quantity}s to be stored in unboxed vectors for improved performance and reduced storage overhead.

\section{\packagename{vector-space} for Linear Algebra}

Users of the \packagename{vector-space} package can find appropriate instances of
the \classname{AdditiveGroup} and \classname{VectorSpace} classes, along with an
instance of \classname{InnerSpace} for dimensionless quantities, in the
\submodule{VectorSpace} module of the \experimentalpackage{} package.

\chapter{Libraries Using \thispackage}

\section{\packagename{atmos} for Modeling Earth's Atmosphere}

The \packagename{atmos} package provides a dimensionally-typed implementation of the 1976 International Standard
Atmosphere, a model of how pressure, temperature, the speed of sound, and related quantities vary with
altitude in the Earth's atmosphere.

\section{\packagename{igrf} for Modeling Earth's Magnetic Field}

The \packagename{igrf} package provides a dimensionally-typed implementation of the International Geomagnetic
Reference Field, including parsing of model text files in the format released by the International Association of
Geomagnetism and Aeronomy and the United States National Oceanographic and Atmospheric Administration. The
package also directly includes the model coeffecients for the most recent model releases.

\chapter{Included Dimensions and Units}

\end{document}
